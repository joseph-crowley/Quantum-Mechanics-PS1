\documentclass[12pt, letterpaper]{article}
\usepackage[margin=1.0in]{geometry}
\usepackage{amsmath}
\usepackage{amssymb}
\usepackage{fancyhdr}
\usepackage{pgfplots}
\usepackage{physics}
\usepackage{wrapfig}
\usepackage{hyperref}
\usepackage{multirow}
\usepackage{amsthm}

\pgfplotsset{compat=1.16}


\renewcommand{\thesubsection}{\thesection\Alph{subsection}}
\renewcommand\qedsymbol{\square}



\title{Particle Physics PS1}
\author{Joe Crowley}
\date{October 2020}

\pagestyle{fancy}
\renewcommand{\headrulewidth}{0pt}
\renewcommand{\footrulewidth}{0pt}

\fancyhf{}
\rhead{
Joe Crowley \\
Physics 215 \\
Problem Set 1
}
\rfoot{Page \thepage}

\begin{document}  

\section{Hermitian Operators}
\subsection{}
\textit{For two Hermitian operators, $\hat{A}, \hat{B},$ both will all eigenvalues positive, show that $\operatorname{Tr}(\hat{A} \hat{B})>0$. }

\begin{equation*}
\begin{split}
    \operatorname{Tr}\left(\hat{A}\hat{B}\right) &= \sum_v \bra{v}\hat{A}\hat{B}\ket{v}\\
    &= \sum_{v,w} \bra{v}\hat{A}\ketbra{w}\hat{B}\ket{v}\\
    &= \sum_{v,w} a_w\braket{v}{w}\bra{w}\hat{B}\ket{v}\\
    &= \sum_{v,w}  a_w b_v\braket{v}{w}\braket{w}{v}\\
    &= \sum_{v,w}  a_w b_v{\left|\braket{v}{w}\right| }^2
\end{split}
\end{equation*}

Since the values $ a_w, b_v, \text{and} {\left|\braket{v}{w}\right| }^2$ are positive, 

$$
    \operatorname{Tr}\left(\hat{A}\hat{B}\right) >0
$$


\subsection{}
\textit{For a Hermitian operator $\hat{A}$ with all eigenvalues positive, show that the matrix elements satisfy the inequality:}
$$
|\langle v|\hat{A}| w\rangle|^{2} \leq\langle v|\hat{A}| v\rangle\langle w|\hat{A}| w\rangle
$$ for any two kets, $|v\rangle$ and $|w\rangle,$ in the Hilbert space.
\begin{proof}
First, let $\ket{\psi}$ be a linear combination of $\ket{v}$ and $\ket{w}$:

\begin{align*}
    \ket{\psi} &= \ket{v} +\alpha \ket{w} \\
    \bra{\psi}&= \bra{v} +\alpha^* \bra{w}
\end{align*}

\begin{align*}
    \bra\psi \hat A \ket\psi&= \left(\bra{v} +\alpha^* \bra{w}\right) \hat A \left(\ket{v} +\alpha \ket{w}\right)\\
    & =\bra v\hat{A} \ket v+|\alpha|^{2} \bra w \hat{A}  \ket w + \alpha \bra  v\hat A \ket w+\alpha^{*}\bra w\hat{A} \ket v
\end{align*}

Let $\alpha = -\frac{\bra{v}\hat A \ket w}{\bra w \hat A \ket w}$.
\begin{align*}
    \bra\psi \hat A \ket\psi& =\bra v\hat{A} \ket v+|\alpha|^{2} \bra w \hat{A}  \ket w + \alpha \bra  v\hat A \ket w+\alpha^{*}\bra w\hat{A} \ket v\\
    & =\bra v\hat{A} \ket v+\left| -\frac{\bra{v}\hat A \ket w}{\bra w \hat A \ket w}\right|^{2} \bra w \hat{A}  \ket w  -\frac{\bra{v}\hat A \ket w}{\bra w \hat A \ket w} \bra  v\hat A \ket w -\frac{\bra{w}\hat A \ket v}{\bra w \hat A \ket w}\bra w\hat{A} \ket v \\
    &= \bra v\hat{A} \ket v - \frac{\left|\bra v \hat A \ket w \right|^2}{\bra w \hat A \ket w}
\end{align*}

Since $\hat A$ has only positive eigenvalues, and each term in the equation above is positive, 

\begin{align*}
     \bra\psi \hat A \ket\psi &\geq 0\\
     \bra v\hat{A} \ket v - \frac{\left|\bra v \hat A \ket w \right|^2}{\bra w \hat A \ket w} &\geq 0\\
     \bra v\hat{A} \ket v &\geq \frac{\left|\bra v \hat A \ket w \right|^2}{\bra w \hat A \ket w}\\
     \bra v\hat{A} \ket v \bra w \hat A \ket w &\geq \left|\bra v \hat A \ket w \right|^2\\
\end{align*}
\end{proof}
    
\subsection{}
\textit{Consider Hermitian operators $\hat{A}, \hat{B}$ and $\hat{H}$ which satisfy the commutations relations, $[\hat{A}, \hat{H}]=[\hat{B}, \hat{H}]=0,$ but $[\hat{A}, \hat{B}] \neq 0 .$ Prove that at least one of the eigenvalues of $\hat{H}$ must be degenerate.}
\begin{proof}
Assume there are no degeneracies in $\hat H$, for basis kets $\ket{\alpha_i}$ and $\ket{\beta_j}$ of operators $\hat A$ and $\hat B$ respectively. This gives
\begin{align*}
    \hat H \ket{\alpha_i} &= a_i \ket{\alpha_i}\\
    \hat H \ket{\beta_j} &= b_j \ket{\beta_j}   . 
\end{align*}
If $\hat H$ is non-degenerate, $\bra{\alpha_i}\ket{\beta_j} = \delta_{ij}$, which implies the eigenkets for $\hat A$ are the same as the eigenkets for $\hat B$. If $\hat A$ and $\hat B$ have the same eigenkets, $[\hat A,\hat B] = 0$. But we assumed $[\hat A,\hat B] \neq 0$. Contradiction!

$\therefore \hat H $ is degenerate.
\end{proof} 


\subsection{}
\textit{For a Hermitian operator $\hat{A}$ show that,
$$
\operatorname{det}\left(e^{\hat{A}}\right)=e^{\operatorname{Tr}(\hat{A})}
$$
For an Hermitian operator $\hat{B}$ with all positive eigenvalues show that,
$$
\ln (\operatorname{det}(\hat{B}))=\operatorname{Tr}(\ln (\hat{B}))
$$
}
$\hat A$ can be diagonalized with matrices $\hat P$ and $\hat{P}^{-1}$ as 
$$
\hat A = \hat{P}^{-1} \hat \Lambda \hat P. 
$$

\begin{align*}
    \exp\left(\hat A\right) & = \sum_{n=0}^\infty \frac{\hat{A}^n} {n!} \\ 
    &= \sum_{n=0}^\infty  \hat{P}^{-1} \frac{\hat{\Lambda}^n} {n!}\, \hat{P}\\
    &= \hat{P}^{-1} \sum_{n=0}^\infty  \frac{\hat{\Lambda}^n} {n!}\, \hat{P}\\
\end{align*}
\begin{align*}
    \operatorname{det}\left(\exp\left(\hat A\right) \right)&= \operatorname{det}\left(\hat{P}^{-1} \sum_{n=0}^\infty  \frac{\hat{\Lambda}^n} {n!}\, \hat{P} \right)\\
    &= \operatorname{det}\left(\hat{P}^{-1}\right)\operatorname{det}\left(\exp\left(\hat \Lambda\right)\right)\operatorname{det}\left(\hat{P}\right)\\
    &= \operatorname{det}\left(\exp\left(\hat \Lambda\right)\right)\\
    &= e^{\lambda_1}e^{\lambda_2}e^{\lambda_3}e^{\lambda_4}\cdots\\
\end{align*}
\begin{align*}
    e^{\operatorname{Tr} \left(\hat A\right)}&=e^{\lambda_1+\lambda_2+\lambda_3+\ldots} \\
    &= e^{\lambda_1}e^{\lambda_2}e^{\lambda_3}e^{\lambda_4}\ldots\\
\end{align*}

\begin{equation*}\boxed{
    \operatorname{det}\left(e^{\hat{A}}\right)=e^{\operatorname{Tr}(\hat{A})}}
\end{equation*}

\begin{align*}
    \operatorname{det}\left(\hat B \right) &= \lambda_1 \lambda_2 \lambda_3 \ldots\\
    \ln\left(\operatorname{det}\left(\hat B \right)\right) & = ln \lambda_1+ ln \lambda_2+ ln \lambda_3+\ldots\\
    \operatorname{Tr}\left( \ln{\hat B} \right) &=  ln \lambda_1+ ln \lambda_2+ ln \lambda_3+\ldots
\end{align*}

\begin{equation*}
\boxed{ \operatorname{Tr}\left( \ln{\hat B} \right) = \ln{\operatorname{det}\left(\hat B\right)}}
\end{equation*}


\section{Anti-Hermitian operators}
\subsection{}
\textit{An operator $\hat{K}$ is said to be anti-Hermitian if it satisfies $\hat{K}^{\dagger}=-\hat{K}$. Show that an anti-Hermitian operator can have at most one real eigenvalue (possibly degenerate).}

\begin{align*}
    \bra{\psi_i} \hat K \ket{\psi_i} &= \lambda_i \\
    \bra{\psi_i} \hat{K}^\dagger \ket{\psi_i} &= \lambda_i^* \\
    -\bra{\psi_i} \hat{K} \ket{\psi_i} &= \lambda_i^* \\
    -\lambda_i &= \lambda_i^* \\
\end{align*}
Only true when $\Re\left(\lambda_i\right)= 0$. The only possible real eigenvalue of any anti-hermitian matrix is 0. 

\subsection{}
\textit{Show that the commutator $[\hat{A}, \hat{B}]$ of two Hermitian operators, $\hat{A}, \hat{B},$ is either anti-Hermitian or zero.}
\begin{proof}
$$
\bra \psi [\hat{A}, \hat{B}] \ket \psi = \bra \psi \hat{A}\hat{B} \ket \psi -\bra \psi \hat{B}\hat{A} \ket \psi 
$$
In the case that $\hat{A}= \hat{B}$, $[\hat{A}, \hat{B}] = 0$. Otherwise, 
\begin{align*}
    [\hat{A}, \hat{B}]^\dagger &= \left(\hat A \hat B \right)^\dagger - \left(\hat B \hat A \right)^\dagger \\
    &=\hat B \hat A -\hat A \hat B\\
    &=-\left(\hat A \hat B-\hat B \hat A  \right)\\
    &= - [\hat{A}, \hat{B}]\\
\end{align*}
$\therefore [\hat{A}, \hat{B}]$ is either Anti-hermitian or 0.  
\end{proof}

\subsection{}
\textit{Demonstrate that it is not possible for two Hermitian operators, $\hat{Q}, \hat{P}$ to satisfy $[\hat{Q}, \hat{P}]=i \hbar \hat{1},$ where $\hat{1}$ denotes the identity operator in the $n$ - dimensional Hilbert space. How do you reconcile this with with the familiar positionmomentum commutation relations?}

$$
\operatorname{Tr}\left(\left[\hat{Q} , \hat{P}\right]\right)=\operatorname{Tr}(\hat{Q} \hat{P})-\operatorname{Tr}(\hat{P} \hat{Q})
$$

Because $\hat P$ and $\hat Q$ are hermitian, 
$$
\left(\hat Q \hat P\right)^\dagger = \hat P \hat Q. 
$$
$$
\implies \operatorname{Tr}(\hat{Q} \hat{P})=\operatorname{Tr}(\hat{P} \hat{Q})
$$
$$
\implies \operatorname{Tr}\left(\left[\hat{Q} , \hat{P}\right]\right) = 0 \quad \forall(\hat Q, \hat P) hermitian
$$

$$ 
\operatorname{Tr}\left(i \hbar \hat 1\right) = i\hbar \dim \mathcal{H}
$$

The equation
$$
 \operatorname{Tr}\left(\left[\hat{Q} , \hat{P}\right]\right) = \operatorname{Tr}\left(i \hbar \hat 1\right)
$$
has no solutions when $ \dim \mathcal{H} \neq 0$. In an infinite dimensional $\mathcal{H}$, $\operatorname{Tr}\left(\hat 1\right)$ is undefined. 


\section{Functions of operators}
\subsection{}
\textit{Consider an operator $\hat{C} \equiv(\hat{A}-\lambda \hat{B})^{-1}$. Assuming that the operator $\hat{A}$ is invertible, with inverse denoted $\hat{A}^{-1}$, derive a formal Taylor series expansion for $\hat{C}$ as a power series in the parameter $\lambda$.}
 
\begin{align*}
\hat C &\equiv \left(\hat A - \lambda \hat B \right)^{-1} \\
& = \left(\hat{A} \hat{A}^{-1} \hat{A}-\hat{A} \hat{A}^{-1} \hat{B} \lambda\right)^{-1}\\
& = \hat{A}^{-1} \left(\hat 1 -  \hat{A}^{-1} \hat{B} \lambda\right)^{-1}\\
& = \hat{A}^{-1} \sum_n \left(\hat{A}^{-1} \hat{B}\right)^n \lambda^n\\
\end{align*}

\begin{equation*}
    \boxed{\hat C= \hat{A}^{-1} \sum_n \left(\hat{A}^{-1} \hat{B}\right)^n \lambda^n}
\end{equation*}

\subsection{}
\textit{Consider an operator $\hat{\mathcal{O}}_{\lambda} \equiv \exp(\lambda \hat{A}+\hat{B})$. What conditions must be placed on the operators $\hat{A}$ and $\hat{B}$ so that $\partial_{\lambda} \hat{\mathcal{O}}_{\lambda}=\hat{A} \hat{\mathcal{O}}_{\lambda} ?$ By performing the Taylor series expansion in powers of $\lambda,$ show that $\operatorname{Tr}\left(\partial_{\lambda} \hat{\mathcal{O}}_{\lambda}\right)=\operatorname{Tr}\left(\hat{A} \hat{\mathcal{O}}_{\lambda}\right)$ for
arbitrary operators $\hat{A}$ and $\hat{B}$.}

\begin{align*}
    \hat{\mathcal{O}}_\lambda &\equiv \exp \left(\lambda \hat{A} + \hat{B} \right)\\
    &= \sum_{n=0}^{\infty} \frac{1}{n !}(\lambda \hat{A}+\hat{B})^{n}\\
    \partial_\lambda \hat{\mathcal{O}}_\lambda &= \sum_{n=1}^{\infty} \frac{1}{(n-1) !}(\lambda \hat{A}+B)^{n-1} \hat{A}\\
    &= \sum_{n=0}^{\infty} \frac{1}{n}(\lambda \hat{A}+\hat{B})^{n} \hat{A}\\
    &= e^{\lambda \hat{A}+\hat{B}} \hat{A} \\
    &= \hat{A} e^{\lambda \hat{A}+\hat{B}} \qquad \text{if} \left[\hat{A},\hat{B}\right] = 0 \\
    &= \hat A \hat{\mathcal{O}}_\lambda
\end{align*}

\begin{equation*}
\boxed{
    \operatorname{Tr}\left(\hat{\mathcal{O}}_\lambda \right) = \operatorname{Tr}\left(\hat A \hat{\mathcal{O}}_\lambda\right)
}
\end{equation*}

\subsection{}
\textit{Consider two operators $\hat{A}$ and $\hat{B}$ whose commutator is a complex number. Show that 
$$\exp(\lambda \hat{A}) \exp(\lambda \hat{B})=\exp \left(\lambda \hat{A}+\lambda \hat{B} \right)  \exp \left(\lambda^{2}[\hat{A}, \hat{B}] / 2\right)$$}
\begin{proof}
    First note that the equation holds for $\lambda = 0$. Then, taking derivatives of each side of the equation, 
    \begin{align*}
        \partial_\lambda\left(\exp(\lambda \hat{A}) \exp(\lambda \hat{B})\right)&=\left(\hat A + \hat B + \lambda [\hat A, \hat B]  \right)\left(\exp(\lambda \hat{A}) \exp(\lambda \hat{B})\right)\\
        \partial_\lambda\left(\exp \left(\lambda \hat{A}+\lambda \hat{B} \right)  \exp \left(\lambda^{2}[\hat{A}, \hat{B}] / 2\right)\right) &= \left(\hat A + \hat B +\lambda[\hat A, \hat B]\right)\left(\exp \left(\lambda \hat{A}+\lambda \hat{B} \right)  \exp \left(\lambda^{2}[\hat{A}, \hat{B}] / 2\right)\right)\\
        \left[\partial_\lambda\left(\exp(\lambda \hat{A}) \exp(\lambda \hat{B})\right) \right]_{\lambda=0} &= \left[\partial_\lambda\left(\exp \left(\lambda \hat{A}+\lambda \hat{B} \right)  \exp \left(\lambda^{2}[\hat{A}, \hat{B}] / 2\right)\right) \right]_{\lambda=0}\\
        \left[{\partial_\lambda}^n\left(\exp(\lambda \hat{A}) \exp(\lambda \hat{B})\right) \right]_{\lambda=0} &= \left[{\partial_\lambda}^n\left(\exp \left(\lambda \hat{A}+\lambda \hat{B} \right)  \exp \left(\lambda^{2}[\hat{A}, \hat{B}] / 2\right)\right) \right]_{\lambda=0}
    \end{align*}
    By induction, 
    \begin{align*}
        \left[{\partial_\lambda}^{n+1}\left(\exp(\lambda \hat{A}) \exp(\lambda \hat{\hat{B}})\right) \right]_{\lambda=0} &=(\hat{A}+\hat{B})\left[{\partial_\lambda}^n \exp (\lambda \hat{A}) \exp (\lambda \hat{B})\right]_{\lambda=0}\\
        & \quad +n[\hat{A}, \hat{B}]\left[{\partial_\lambda}^{n - 1} \exp (\lambda \hat{A}) \exp (\lambda \hat{B})\right]_{\lambda=0}\\
        &= (\hat{A}+\hat{B})\left[{\partial_\lambda}^{n } \exp \left(\lambda \hat{A}+\lambda \hat{B}+\frac{\lambda^{2}}{2}[\hat{A}, \hat{B}]\right)\right]_{\lambda=0}\\
        & \quad+n[\hat{A}, \hat{B}]\left[{\partial_\lambda}^{n-1 } \exp \left(\lambda \hat{A}+\lambda \hat{B}+\frac{\lambda^{2}}{2}[\hat{A}, \hat{B}]\right)\right]_{\lambda=0}\\
        &= \left[{\partial_\lambda}^{n }(\hat{A}+\hat{B}+\lambda[\hat{A}, \hat{B}]) \exp \left(\lambda \hat{A}+\lambda \hat{B}+\frac{\lambda^{2}}{2}[\hat{A}, \hat{B}]\right)\right]_{\lambda=0} \\ 
        &= \left[{\partial_\lambda}^{n + 1} \exp \left(\lambda \hat{A}+\lambda \hat{B}+\frac{\lambda^{2}}{2}[\hat{A}, \hat{B}]\right)\right]_{\lambda=0}
    \end{align*}
    since the equation and each term in its expansion agree at $\lambda = 0$, 
    $$
        \boxed{\exp(\lambda \hat{A}) \exp(\lambda \hat{B})=\exp \left(\lambda \hat{A}+\lambda \hat{B} \right)  \exp \left(\lambda^{2}[\hat{A}, \hat{B}] / 2\right)}
    $$
\end{proof}


\subsection{}
\textit{Find the eigenvalues of the three operators $\hat{A}, \hat{B}$ and $\hat{C}$ which satisfy $\hat{A}^{2}=p^{2}, \hat{B}^{2}=p \hat{B}$ and $\hat{C}^{3}=p^{2} \hat{C}$ for real c-number $p$.}

\begin{align*}
    \hat{A}^2 \ket a = \hat A \left(a \ket a\right)  &= p ^2 \ket a \\
    &\implies a = \pm p \\
    \hat{B}^2 \ket b = \hat{B} \left(b \ket b \right)  &= p \hat B \ket b \\
     &\implies b = p \\
    \hat{C}^3 \ket c=\hat{C} c^2 \ket c  &= p^2 \hat{C} \ket c \\
     &\implies c = \pm p \\
\end{align*}

\end{document}
